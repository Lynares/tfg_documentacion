\chapter*{}
%\thispagestyle{empty}
%\cleardoublepage

%\thispagestyle{empty}

\input{portada/portada_2}



\cleardoublepage
\thispagestyle{empty}

\begin{center}
{\large\bfseries \myTitle}\\
\end{center}
\begin{center}
%{\myName}\\
\myName\\
\end{center}

%\vspace{0.7cm}
\noindent{\textbf{Palabras clave}: identificación de tráfico, Monitor de Red, clasificación de tráfico, emparejamiento de flujos, Bro}\\

\vspace{0.7cm}
\noindent{\textbf{Resumen}}\\
La identificación de tráfico en red consiste en asignar instancias del tráfico circulante en la red a la aplicación o tipo de aplicación que genera dicho tráfico. Resulta de gran interés desde el punto de vista de la gestión de red ya que permite hacer ingeniería de tráfico y está también relacionada con la monitorización de la seguridad.
\intro Una de las técnicas más fiables para categorizar el tráfico se basa en la denominada Inspección Profunda de Paquetes, DPI, que analiza el contenido de los paquetes en busca de patronos que se puedan asociar a las diferentes aplicaciones. Consecuentemente, presenta problemas de privacidad, por acceder al contenido, y de escalabilidad, por su alto coste computacional.
\intro En trabajos de investigación desarrollados en el Departamento de Teoría de la Señal, Telemática y Comunicaciones de la Universidad de Granada se ha propuesto y evaluado una técnica que, a partir del emparejamiento de flujos mediante una sencilla función de similitud, permite reducir el impacto de ambos problemas.
\intro En el presente trabajo, se pretende incorporar la técnica de emparejamiento en un sistema de monitorización de red para posibilidar su uso en redes en explotación. Así, se aborda la creación de un módulo que implementa la técnica mencionada en Bro, uno de los sistemas de monitorización de amplio uso y gran flexibilidad. El módulo desarrollado se evalúa sobre trazas reales capturadas en la red con resultados satisfactorios.
\cleardoublepage


\thispagestyle{empty}


\begin{center}
{\large\bfseries Implementation on real time of traffic network identification systems}\\
\end{center}
\begin{center}
Álvaro Maximino Linares Herrera\\
\end{center}

%\vspace{0.7cm}
\noindent{\textbf{Keywords}: traffic identification, Network Monitoring System, traffic classification, flow pairing, Bro}\\

\vspace{0.7cm}
\noindent{\textbf{Abstract}}\\

The identification of network traffic involves assigning traffic instances flowing in the net to the application or kind of application generated by that traffic. It is of great interest from the network´s management point of view since it allows to create traffic engeneering and is also related to the safety monitorization.

\intro One of the most trustworthy techniques to categorize the traffic is based on the Deep Package Inspection, DPI, which analyses the content of those packages seeking for patterns than can be linked to the different applications. Consequently, it owns privacy problems due to the content access and escalability for his high computational cost.

\intro In some research works, carried by the Department of Theory of the Signal, Telematics and Communications of the University of Granada, a technique has been proposed and evaluated which links the flow pairs through a simple similarity function which allows to reduce the impact of both problems.

\intro In the current document, it is expected to embody the flow pairing technique in a Network Monitoring System to enable its usage in explotation networks. This way, it is faced the creation of a module that implements the procedure above mentioned in Bro, one of the monitorization systems of broad use and great flexibility. The module developed is evaluated over real traces captured in the network with satisfactory outcomes.

\chapter*{}
\thispagestyle{empty}

\noindent\rule[-1ex]{\textwidth}{2pt}\\[4.5ex]

Yo, \textbf{\myName}, alumno de la titulación TITULACIÓN de la \textbf{Escuela Técnica Superior
de Ingenierías Informática y de Telecomunicación de la Universidad de Granada}, con DNI 76669401M, autorizo la
ubicación de la siguiente copia de mi Trabajo Fin de Grado en la biblioteca del centro para que pueda ser
consultada por las personas que lo deseen.

\vspace{6cm}

\noindent Fdo: \myName

\vspace{2cm}

\begin{flushright}
Granada a 11 de septiembre de 2017.
\end{flushright}


\chapter*{}
\thispagestyle{empty}

\noindent\rule[-1ex]{\textwidth}{2pt}\\[4.5ex]

D. \textbf{\myProf}, Profesor del Departamento Teoría de la Señal, Telemática y Comunicaciones (TSTC) de la Universidad de Granada.

\vspace{0.5cm}

%D. \textbf{Nombre Apellido1 Apellido2 (tutor2)}, Profesor del Área de XXXX del Departamento YYYY de la Universidad de Granada.


%\vspace{0.5cm}

\textbf{Informa:}

\vspace{0.5cm}

Que el presente trabajo, titulado \textit{\textbf{\myTitle}},
ha sido realizado bajo su supervisión por \textbf{\myName}, y autorizamos la defensa de dicho trabajo ante el tribunal
que corresponda.

\vspace{0.5cm}

Y para que conste, expiden y firman el presente informe en Granada a 11 de septiembre de 2017.

\vspace{1cm}

\textbf{Los directores:}

\vspace{5cm}

\noindent \textbf{\myProf}

\chapter*{Agradecimientos}
\thispagestyle{empty}

       \vspace{1cm}

En primer lugar, dar las gracias a mi maravillosa familia, tanto a los que están, cómo a los que se han ido a lo largo de los años, por su confianza y cariño, así como darme la posibilidad de realizar estos estudios.
\intro También dar las gracias a Jesús, por su dedicación e inspiración, así como su paciencia con un alumno como yo.
\intro Por último, dar las gracias a mis amigos, por su apoyo incondicional. Y en especial a Carmen y a Javi.
