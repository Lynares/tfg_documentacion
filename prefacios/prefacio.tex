\chapter*{}
%\thispagestyle{empty}
%\cleardoublepage

%\thispagestyle{empty}

\input{portada/portada_2}



\cleardoublepage
\thispagestyle{empty}

\begin{center}
{\large\bfseries \myTitle}\\
\end{center}
\begin{center}
%{\myName}\\
\myName\\
\end{center}

%\vspace{0.7cm}
\noindent{\textbf{Palabras clave}: inspección profunda de paquetes, bro, flujos, identificación, clasificación, red, tráfico}\\

\vspace{0.7cm}
\noindent{\textbf{Resumen}}\\

La identificación de tráfico en red es realmente importante 
para aplicaciones de ingeniería de tráfico y de seguridad. 
\intro
En este trabajo se tratará la creación de un programa para 
un NMS (Network Monitoring System), en este caso se usará 
BRO, mediante el cual se pueda resolver el emparejamiento 
de flujos. BRO consiste en un NMS que funciona mediante el 
terminal en Linux o Mac, una de las peculiaridades de este 
programa es que para la creación de scripts que nos permitan 
extender la funcionalidad de la que dispone, tendremos que usar 
BRO como lenguaje de programación. Es un lenguaje de scripting, 
el cual está orientado a eventos, que se lanzan cuando ocurre 
algo relacionado con el control y análisis de redes, es un 
lenguaje que para los que vienen de C++, Java o Python, no debe 
de suponer un gran reto, más allá de acostumbrarse a sus 
sintaxis. Es un lenguaje potente que al estar orientado a redes 
nos permite obtener mucha información de los flujos que tenemos 
en la red o en el archivo que vayamos a analizar. En este trabajo 
mediante implementaciones offline se verificará la eficacia de 
esta técnica de clasificación de tráfico. 
\cleardoublepage


\thispagestyle{empty}


\begin{center}
{\large\bfseries Project Title: Project Subtitle}\\
\end{center}
\begin{center}
First name, Family name (student)\\
\end{center}

%\vspace{0.7cm}
\noindent{\textbf{Keywords}: Keyword1, Keyword2, Keyword3, ....}\\

\vspace{0.7cm}
\noindent{\textbf{Abstract}}\\

Write here the abstract in English.

\chapter*{}
\thispagestyle{empty}

\noindent\rule[-1ex]{\textwidth}{2pt}\\[4.5ex]

Yo, \textbf{\myName}, alumno de la titulación TITULACIÓN de la \textbf{Escuela Técnica Superior
de Ingenierías Informática y de Telecomunicación de la Universidad de Granada}, con DNI 76669401M, autorizo la
ubicación de la siguiente copia de mi Trabajo Fin de Grado en la biblioteca del centro para que pueda ser
consultada por las personas que lo deseen.

\vspace{6cm}

\noindent Fdo: \myName

\vspace{2cm}

\begin{flushright}
Granada a 3 de septiembre de 2017.
\end{flushright}


\chapter*{}
\thispagestyle{empty}

\noindent\rule[-1ex]{\textwidth}{2pt}\\[4.5ex]

D. \textbf{\myProf}, Profesor del Área de XXXX del Departamento YYYY de la Universidad de Granada.

\vspace{0.5cm}

%D. \textbf{Nombre Apellido1 Apellido2 (tutor2)}, Profesor del Área de XXXX del Departamento YYYY de la Universidad de Granada.


%\vspace{0.5cm}

\textbf{Informa:}

\vspace{0.5cm}

Que el presente trabajo, titulado \textit{\textbf{\myTitle}},
ha sido realizado bajo su supervisión por \textbf{\myName}, y autorizamos la defensa de dicho trabajo ante el tribunal
que corresponda.

\vspace{0.5cm}

Y para que conste, expiden y firman el presente informe en Granada a 3 de septiembre de 2017.

\vspace{1cm}

\textbf{Los directores:}

\vspace{5cm}

\noindent \textbf{\myProf \ \ \ \ \ Nombre Apellido1 Apellido2 (tutor2)}

\chapter*{Agradecimientos}
\thispagestyle{empty}

       \vspace{1cm}


Poner aquí agradecimientos...

