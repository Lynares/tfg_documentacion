\chapter{Uso del módulo creado}\label{cap.uso}

En este apartado se encontrará una pequeña guía de los comandos de Bro, así como una pequeña guía del uso del módulo creado.

\section{Comandos de Bro}

A continuación, se encontrará una lista de los comandos más interesantes de Bro.

\begin{lstlisting}[style=Consola]
    -b|--bare-mode                 | no carga los scripts del directorio base/
    -d|--debug-policy              | activa el modo debug
    -f|--filter <filter>           | filtro tcpdump
    -h|--help|-?                   | ayuda
    -i|--iface <interface>         | lee de la interfaz dada
    -r|--readfile <readfile>       | lee el archivo dado
    -s|--rulefile <rulefile>       | lee las reglas del archivo dado
    -t|--tracefile <tracefile>     | activa la ejecucion de trazas 
    -w|--writefile <writefile>     | escribe en el archivo dado
    -x|--print-state <file.bst>    | muestra el contenido del archivo de estado
    -C|--no-checksums              | ignaora checksums
    -F|--force-dns                 | fuerza DNS
    -N|--print-plugins             | muestra los plugins disponibles y sale
    -Q|--time                      | muestra el tiempo de ejecucion en stderr
    -R|--replay <events.bst>       | repite los eventos
    -X|--broxygen <cfgfile>        | genera la documentacion basada en el archivo de configuracion dado   

\end{lstlisting}

\section{Uso de Bro}

Para usar Bro se tendrá que hacer lo siguiente.

\begin{lstlisting}[style=Consola]
	bro [opciones] [archivos]
\end{lstlisting}

\intro Pero antes de la ejecución de Bro, se tendrá que fijar el \textit{PATH} en la carpeta de Bro.

\begin{lstlisting}[style=Consola]
	export PATH =/usr/local/bro/bin:$PATH
\end{lstlisting}

\intro Una vez realizado este paso, se podrá ejecutar el módulo creado, de la siguiente forma.

\begin{lstlisting}[style=Consola]
	bro -b -r pcap/nitroba.pcap scripts/bro-flows/broflows.bro
\end{lstlisting}

