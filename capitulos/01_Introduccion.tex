\chapter{Introducción}

La clasificación de tráfico en red es una tarea importante 
en lo relativo a las comunicaciones, en un mundo cada vez 
más digitalizado e intercomunicado, lo que más importa es 
la seguridad y para ello es esencial la clasificación del 
tráfico, permitiendo detectar de forma temprana intrusiones 
y comportamientos anormales, para los cuales podremos 
prepararnos, de esta forma, por ejemplo, el encargado de un 
servidor podrá mantener la calidad de servicio, pues mediante 
ISP (Internet Service Provider) se puede establecer diferentes 
niveles de prioridad en el tráfico de red.
\intro
En este trabajo haremos uso de BRO, un NMS (Network Monitoring System), 
al cual mediante la implementación de técnicas de emparejamiento de 
flujos, lo dotaremos de la capacidad analítica de discernir que flujos 
son emparejables, y por lo tanto pertenecen a lo mismo.

\section{Conceptos básicos}

En este apartado vamos a aclarar los conceptos básicos que son necesarios 
para entender este trabajo.
\intro
Lo primero que tenemos que tener claro es lo que es un paquete. Un paquete 
es cada un fragmento de la información que queremos enviar a través de una red. 
Este paquete contiene datos sobre quiénes somos, digitalmente, y hacia donde 
enviamos la información, así que incluso un sólo paquete da mucha información 
sobre quienes somos en la red.
\intro
Lo siguiente que debemos de saber es que es un flujo, para ello tenemos que 
volver hacia la definición anterior, pues consiste en la historia de un grupo 
de paquetes, esto quiere decir cómo se mueven los paquetes a través de las capas 
de red, de datos y la física. Para quien no sepa sobre redes se estará 
preguntando que son estas capas, pues bien, estas capas son del modelo OSI, que 
es el modelo de referencia para los protocolos de red, siendo las distintas 
capas las siguientes:
\intro
1. Aplicación
2. Presentación
3. Sesión
4. Transporte
5. Red
6. Enlace de datos
7. Física
\intro
Lo siguiente que explicaremos será que es un NMS o Network Monitoring System, 
como el nombre nos indica consiste en un programa que se dedica a monitorizar 
el tráfico del sistema y en caso de encontrar algún problema avisará al 
administrador del sistema. En nuestro caso el NMS que usaremos será Bro, 
aunque existen otros.

\section{TCP y UDP}

Describir que es TCP y que UDP, poner imágenes del libro sobre la comunicación.