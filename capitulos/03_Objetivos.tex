\chapter{Objetivos}
En este trabajo lo que se trata de hacer es:
\begin{itemize}
\item Demostrar que es posible analizar el tráfico mediante emparejamiento de flujo.
\item Realizar un programa que sea capaz, a partir de un archivo pcap, analizar los flujos.
\item Que dicho programa muestre la información de los flujos emparejado.
\item Implementación de la función de la comparación que se halla en el ensayo [1].
\end{itemize}

Para demostrar que es posible analizar el tráfico mediante emparejamiento 
de flujo, y por tanto con DPI, se usará Bro para gestionar el tráfico de 
la red, aunque al ser una implementación offline lo que se pretende realizar 
tendrá que leer datos de un archivo pcap.
\intro
Al realizar una extensión para Bro tendremos que adaptarnos y aprender a 
programar en el lenguaje que utilizan para ello, Bro, el cual tiene ciertas 
peculiaridades, aunque como ya sabemos programar en distintos lenguajes, el 
aprendizaje de este lenguaje es muy rápido, pues la gran mayoría de las cosas 
ya sabemos hacerlo, sólo falta adaptarse a las peculiaridades que tiene, como 
cualquier lenguaje. Alguna de estas peculiaridades es por ejemplo el acceso a 
los datos dentro de un tipo, que se realiza mediante \$, cuando en otros 
lenguajes accedemos a ellos mediante . ó ->.
\intro
Para mostrar la información de los flujos será necesario crear una función 
que nos muestre por pantalla que contienen dichos flujos. Realizar la función 
es para no repetir código, lo cual entra dentro de las buenas prácticas de 
programación. En dicha función se accederemos a los datos de IP origen y 
destino y los puertos origen y destino de los flujos que estamos comparando, 
esta función no realiza comparaciones, simplemente muestra la información, 
pues esta función es llamada después de encontrar dos flujos que son emparejables.
\intro
Para ver si dos flujos son emparejables, primero tendremos que ver si las IP’s 
y los puertos de origen y destino son iguales. En caso de serlo pasaremos a 
aplicar la función, que se encuentra en el primer apartado, la cual nos devolverá 
un número que compararemos con un umbral que nosotros definamos, haciendo que 
si es mayor no sean emparejables y si es menor que dicho umbral si sean emparejables.