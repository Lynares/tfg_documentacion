\chapter{Conclusiones y trabajo futuro}

\section{Conclusiones}

Después de lo visto anteriormente se puede llegar a 
la conclusión de que mediante un NMS, al cual mediante 
la agregación de funcionalidades, se le proporciona la 
capacidad de clasificar el tráfico de una red mediante 
emparejamiento de flujos. En nuestro caso los protocolos 
analizados son solamente \textbf{TCP y UDP}, es decir, 
los protocolos de la capa de transporte y en algunos casos 
también el protocolo ICMP.
\intro
Trabajar con Bro al principio fue algo tedioso, pues 
había que aprender un lenguaje de programación propio de la empresa 
y no había mucha documentación disponible, a parte de 
la que hay en su página web, pero una vez realizadas 
las primeras tomas de contacto y resueltos los problemas 
derivados de no conocer bien el lenguaje, problemas ya expuestos 
en anteriores apartados todo fue bastante sencillo.
\intro
Fue muy sencillo poder mostrar los datos de los flujos, 
pues que se disponga de un tipo de dato que almacene todos 
los datos correspondientes al flujo hace que trabajar con 
ello sea muy fácil, incluso mostrar el de todos los 
flujos que pasan por el programa, y que el rendimiento 
no se vea afectado. Puede considerarse un inconveniente 
que no disponga de interfaz, pues en el terminal se puede 
llegar a confundir las cosas dependiendo de la velocidad 
con la que saque la información, que sea un lenguaje basado 
en eventos también puede llegar a ser un problema a la hora 
de mostrar la información, pues puede ser que estemos 
mostrando la información de dos flujos que estamos comparando 
de tipo UDP por ejemplo, y que salte un evento de TCP y 
se muestre mientras todavía se está mostrando la información 
del evento de UDP.

\section{Trabajo futuro}

Obviamente este programa al solo poder emparejar flujos 
mediante el protocolo de transporte se puede quedar corto 
a la hora de querer obtener un análisis más exhaustivo, de 
modo que inspeccionemos los paquetes de una forma más profunda, 
de modo que a este programa se le puede agregar más funcionalidades 
haciendo que sea más completo, y clasifiquen además los flujos 
por el tipo de protocolo en la capa de aplicación pertenecen, 
con esto nos referimos a HTTP, SMTP, FTP, DNS y demás. Esto 
podría llegar a realizarse analizando más detenidamente los puertos. 
Esta extensión se podría realizar con algunos clasificadores de puertos 
que ya están realizados y se pueden encontrar con cierta facilidad en 
GitHub.
