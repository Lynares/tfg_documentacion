\chapter{Diseño y arquitectura del sistema}

En este capítulo se encuentra cómo se pretende resolver el problema que ha sido presentado. Por lo tanto 
se encuentran detalles de la arquitectura del sistema. Detalles del módulo y sus funciones. La gestión de 
los flujos activos y emparejados.

\intro Todo esto es contado desde la vista del diseño, por lo tanto aquí no se encontrará nada de código.

\section{Arquitectura del sistema}

En este problema la arquitectura que se va a emplear es una arquitectura modular. En dicha arquitectura el software 
se divide en grupos que son bien diferenciados, pero a la vez están muy bien acoplados los unos con los otros.

\intro Bro dispone además de varios \textit{frameworks}. Con estos \textit{frameworks} se podrá crear módulos muy 
potentes. Algunas de las utilidades de estos \textit{frameworks} son las siguientes.
\begin{itemize}
\item Geolocalización. Se podrá encontrar la localización geográfica de una IP.
\item Análisis de ficheros. 
\item Framework de loggins. Con este \textit{framework} se podrá extender los archivos de registro que se generan.
\item NetControl framework. Este \textit{framework} permitirá tener un control muy amplio y diverso del tráfico.
\end{itemize}

\intro Para mayor conocimiento de estos \textit{frameworks} y otros acceda a \cite{broframeworks}.

\intro Se podría pensar que para el módulo que se requiere lo ideal sería usar el \textit{framework} de NetControl, 
pero no es así. Es un \textit{framework} que aunque se extienda no aporta nada que no tenga ya Bro para la creación 
del módulo que se necesita.

\intro Por lo tanto la arquitectura final del sistema que se pretende construir será un módulo que sea compatible 
con el resto de módulos de los que está compuesto Bro.

\section{Módulo y funciones}

El módulo que se pretende construir debe de constar solamente con lo imprescindible para que pueda 


\section{Gestión de flujos}



\section{Estructura de datos}
