\chapter{Diseño y arquitectura del sistema}

En este capítulo se encuentra cómo se pretende resolver el problema que ha sido presentado. Por lo tanto 
se encuentran detalles de la arquitectura del sistema. Detalles del módulo y sus funciones. La gestión de 
los flujos activos y emparejados.

\intro Todo esto es contado desde la vista del diseño, por lo tanto aquí no se encontrará nada de código.

\section{Arquitectura del sistema}

En este problema la arquitectura que se va a emplear es una arquitectura modular. En dicha arquitectura el software 
se divide en grupos que son bien diferenciados, pero a la vez están muy bien acoplados los unos con los otros.

\intro Bro ya está divido en módulos que funcionan como un único módulo. También dispone de algunos módulos 
que pueden ser sustituidos, eliminados y simplemente desactivados cuando se vaya a realizar un análisis y 
todo el sistema seguirá funcionando bien. Por lo tanto el módulo que se vaya a implementar debe de ser independiente 
de estos módulos que pueden ser desactivados, aunque también debería de poder hacer uso de ellos en determinadas 
tareas especificas.

\section{Módulo y funciones}

\section{Gestión de flujos}

\section{Estructura de datos}
