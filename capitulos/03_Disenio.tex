\chapter{Diseño y arquitectura del sistema}

En este capítulo se encuentra cómo se pretende resolver el problema que ha sido presentado. Por lo tanto 
se encuentran detalles de la arquitectura del sistema. Detalles del módulo y sus funciones. La gestión de 
los flujos activos y emparejados.

\intro Todo esto es contado desde la vista del diseño, por lo tanto aquí no se encontrará nada de código.

\section{Arquitectura del sistema}

En este problema la arquitectura que se va a emplear es una arquitectura modular. En dicha arquitectura el software 
se divide en grupos que son bien diferenciados, pero a la vez están muy bien acoplados los unos con los otros.

\intro Bro dispone además de varios \textit{frameworks}. Con estos \textit{frameworks} se podrá crear módulos muy 
potentes. Algunas de las utilidades de estos \textit{frameworks} son las siguientes.
\begin{itemize}
\item Geolocalización. Se podrá encontrar la localización geográfica de una IP.
\item Análisis de ficheros. 
\item Framework de loggins. Con este \textit{framework} se podrá extender los archivos de registro que se generan.
\item NetControl framework. Este \textit{framework} permitirá tener un control muy amplio y diverso del tráfico.
\end{itemize}

\intro Para mayor conocimiento de estos \textit{frameworks} y otros acceda a \cite{broframeworks}.

\intro Se podría pensar que para el módulo que se requiere lo ideal sería usar el \textit{framework} de NetControl, 
pero no es así. Es un \textit{framework} que aunque se extienda no aporta nada que no tenga ya Bro para la creación 
del módulo que se necesita.

\intro Por lo tanto la arquitectura final del sistema que se pretende construir será un módulo que sea compatible 
con el resto de módulos de los que está compuesto Bro.

\section{Módulo y funciones}

El módulo que se pretende construir deberá de contar solamente con lo mínimo. De esta forma se pretende que su 
ejecución sea óptima. Una mala programación del módulo dejará al analizador de red ejecutando el análisis de forma 
indefinida. Para que sea óptimo la funcionalidad del módulo deberá de estar clara y presente siempre.

\intro Las funciones que se pretende que tenga este módulo en esencia son dos, la detección y almacenado 
del tráfico y la aplicación de la fórmula a los distintos flujos que se han detectado. De una forma más amplia 
las funcionalidades del módulo serán las siguientes. 

\begin{itemize}
\item \textit{Función que aplique la fórmula de emparejamiento}. 
\intro A esta función se le pasará dos flujos, de forma que se aplique la fórmula y se discierna si pueden 
ser emparejados o no.
\item \textit{Funciones que detecten el tráfico}. 
\intro Esto se hará con los eventos de Bro. Los eventos detectarán el tipo de tráfico que se está analizando 
y aplicarán la función anterior.
\end{itemize}

\intro Para realizar esto es necesario conocer como gestiona los flujos Bro y de que forma se mantendrán los que 
son emparejados y los que están activos.

\section{Gestión de flujos}

La gestión de flujos en Bro pasa completamente por eventos. Por lo tanto se tendrá que crear variables globales 
para el almacenamiento de los flujos que sean potencialmente emparejables. 

\intro El nacimiento de un flujo es controlado por un evento. Por lo tanto cuando se detecta un nuevo flujo se 
lanza un evento. Este evento tendrá que ser usado para controlar si ya se tiene un flujo de características 
parecidas o por el contrario no. Esta decisión tendrá distintos efectos en el módulo. De no tenerlo detectado 
se tendrá que almacenar como un flujo activo nuevo. Si se tiene almacenado se deberá de comparar con los flujos 
activos almacenados que pueden ser emparejados.

\intro La muerte de un flujo también es controlada por eventos. Aquí lo importante es, si todavía es interesante 
a pesar de que este muerto, tener el flujo almacenado. Si no se quiere tener el flujo almacenado porque ya no 
está activo en el sistema, tendrá que ser almacenado en el mismo evento. Si se quiere almacenar porque puede 
seguir siendo útil para las comparaciones que se pueden realizar en el futuro, no se hará nada. 



\section{Estructura de datos}
