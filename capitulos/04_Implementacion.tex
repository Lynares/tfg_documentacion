\chapter{Implementación}

Ahora vamos a proceder a explicar cómo se ha solucionado los problemas 
propuestos, siendo estos explicados a través del programa que se ha 
realizado, quedando así más clara la explicación.
\intro
\noindent Lo primero será mostrar que estructura se ha utilizado para almacenar los 
flujos.
\intro
\begin{lstlisting}[style=CodigoC]

## Tablas para guardar los flujos que son emparejados
global collection: table[addr, addr, port, port] of vector of connection &synchronized;
global collection_added: table[addr, addr, port, port] of vector of connection;

## El umbral: "Comparar la constante 'k', que es el umbral que fijare con el resultado que devuelve la función,
## si es mas grande el resultado que 'k' se puede decir que los dos flujos son iguales, si es mas pequenio podemos decir que los dos flujos no son iguales"
## resultado del umbral que calculamos
global umbral: double;

## Definimos el umbral, de manera global para hacer las comparaciones
global k=10;

\end{lstlisting}

\noindent En este primer código tenemos que: 
\begin{itemize}
\item Vamos a guardar los flujos en una tabla \textit{collection} que tiene como índices 
dos direcciones IP y dos puertos, y a partir de estos indices tendremos un \textit{vector de connection}.
\item La tabla \textit{collection\_added} está destinada para guardar los flujos que son 
emparejados.
\item El \textit{umbral}, que será una variable global, y será donde almacenemos el resultado 
de la función de comparación.
\item \textit{k} es el umbral que definimos nosotros al principio, que será el que usemos como 
referencia para saber si dos flujos son emparejables o no.
\end{itemize}

\begin{lstlisting}[style=CodigoC]
## funcion para la comparacion de los flujos, c1 el flujo que esta el primero en el vector de la tabla y c2 para el flujo que es candidato a ser emparejado

function emparejamiento(c1: connection, c2: connection ):double {

  ## Añadimos variables para comprobar en la tabla, sin hacer bucle
  local orig = c1$id$orig_h;
  local dest = c1$id$resp_h;
  local po = c1$id$orig_p;
  local pd = c1$id$resp_p;

  local Nip = |collection[orig,dest,po,pd]|; ## Variable para saber cuantas conexiones tenemos
  local Po1: count; ## Puerto origen del primer flujo
  local Po2: count; ## Puerto origen del segundo flujo
  local Pd1: count; ## Puerto destino del primer flujo
  local Pd2: count; ## Puerto destino del segundo flujo
  local k1 = 1;  ## Variable fija
  local k2 = 10; ## Variable fija
  local dt: double; ## Variable para la diferencia de los tiempos
  local resultado = 0.0; ## Lo iniciamos a 0
  
  print c1$uid;
  print c2$uid;

  print fmt("Numero de Nip en table: %d", Nip);
  informacion_coincidencia(c1,c2);
  print fmt("Tiempo de inicio del flujo: %s", |c1$start_time|);
  print fmt("Tiempo de inicio del flujo: %s", |c2$start_time|);
  ## Para dp1 y dp2 que son 1-norm usamos la "Manhattan norm" que dice lo siguiente: SAD(x1,x2) = sumatoria(x1i - x2i)
  ## k1 y k2 son dos variables que nosotros le ponemos de forma manual, en este caso las pondremos como locales con 1 y 10 respectivamente
  ## dt es la diferencia de tiempo entre los time stamp de los primeros flujos de los flujos
  ## el tipo time se supone que es como un double, por lo tanto podremos restarlos sin problemas
  ## para la comparacion de puertos primero tendremos que hacer uso de la funcion  port_to_count [https://www.bro.org/sphinx/scripts/base/bif/bro.bif.bro.html#id-port_to_count]
  ## la cual nos pasa el puerto, que recordamos que va tambien con un string en el cual se nos dice que tipo es, a un
  ## valor numerico que si podremos restar sin problemas
  ## La funcion quedaria asi: (Nip-1)+(1/(dp1+k1))+(1/(dp2+k1))+(1/(dt+k2))
  Po1=port_to_count(c1$id$orig_p);
  Pd1=port_to_count(c1$id$resp_p);
  Po2=port_to_count(c2$id$orig_p);
  Pd2=port_to_count(c2$id$resp_p);
  ## local t1: double;
  ## local t2: double;
  ## t1 = time_to_double(c1$start_time);
  ## t2 = time_to_double(c2$start_time);

  dt=(|c1$start_time| - |c2$start_time|);

  ## print fmt("Tiempo paquete 1: %s", t1);
  ## print fmt("Tiempo paquete 2: %s", t2);
  print fmt("Diferencia de tiempo: %s", dt);

  resultado=(Nip-1)+(1/((Po1-Po2)+k1))+(1/((Pd1-Pd2)+k1))+(1/(dt+k2));

 return resultado;

}
\end{lstlisting}

\noindent En este trozo de código se muestra la función que se usa para devolver 
el resultado de la función de comparación. El resultado siempre es comparado 
con el umbral que hemos definido antes de lanzar el programa, y una vez que se 
usa esta función quiere decir que son \textit{comparables}, pero no sabemos todavía si son 
\textit{emparejables}. Si son emparejables lo sabremos solamente después de 
que hayamos ejecutado la función y se realice la comparación con el umbral. Para que 
quede más claro aquí esta el código de la comparación con el umbral.

\begin{lstlisting}[style=CodidoC]
. . .

	umbral=emparejamiento(cl,c);
    print fmt("connection_finished");
    if(umbral>k){
      ## Si el umbral calculado es mayor que el umbral de comparacion lo añadimos
      print fmt("Si son emparejables TCP"); ## Mostramos TCP para saber en que evento se han calculado
      collection[orig,dest,po,pd][|collection[orig,dest,po,pd]|] = c;
      informacion_coincidencia(cl, c);
      if( [orig,dest,po,pd] !in collection_added ){

        collection_added[orig,dest,po,pd]=vector(c);
        print fmt("Añadimos una nueva conexion al vector de coincidencias");
      } else {

        ## Si ya esta, lo añadimos
        collection_added[orig,dest,po,pd][|collection_added[orig,dest,po,pd]|] = c;
        print fmt("Ya esta en vector de coincidencias y la añadimos");
      }

    } else{
      ## Si el umbral calculado es menor que el umbral de comparacion no lo añadimos
      print fmt("No son emparejables TCP");

    }
    
. . .
\end{lstlisting}

\noindent En este ejemplo se calcularía para las conexiones de tipo \textit{TCP}.
\intro
La comparación se dará cuando Bro reconozca un flujo de un determinado tipo 
de protocolo, en nuestro caso será mediante los eventos destinados a 
reconocer los tipos de protocolos de la \textit{capa de transporte}.
Dichos eventos son:

\begin{itemize}
\item connection\_established, el cual se lanza cuando Bro localiza un SYN-ACK que corresponda a un handshake de TCP.
\item connection\_finished, el cual lanza Bro cuando una conexión TCP finaliza de forma normal. 
\item udp\_request, el cual se lanza cuando Bro localiza un flujo que corresponde con UDP lanzado desde el origen.
\item udp\_reply, que es lanzado por Bro cuando localiza una respuesta a un flujo del anterior evento.
\item icmp\_echo\_request, lanzado por Bro cuando localiza un flujo de tipo ICMP de echo.
\item icmp\_echo\_reply, el cual es lanzado por Bro cuando localiza una respuesta a los flujos del evento anterior.
\end{itemize}

\noindent Dentro de estos eventos descritos se realiza la comprobación de que dos 
flujos se pueden comparar, ya que tienen los mismos IP’s y puertos de 
origen y destino. Una vez en la función de comparación se almacenarán 
dichos puertos, pasándolos al tipo count para su uso en la función, se 
establecerán dos constantes, se calculará la diferencia de tiempo y en 
una variable se almacenará el número de flujos que hay con esos datos. 
Cuando realicemos esto, aplicaremos la función de cálculo y podremos 
decir si los flujos son emparejables.
\intro
Para mostrar la información de los flujos tenemos dos opciones, de las 
cuales mostramos el código a continuación:
\intro
\begin{lstlisting}[style=CodigoC]

## Creo funcion auxiliar para ver la informacion del flujos que son coincidentes

function informacion_coincidencia(c: connection, p: connection){
    print fmt("Informacion del primer flujo  IPo: %s , Po: %s , IPd: %s , Pd: %s ", c$id$orig_h, c$id$orig_p, c$id$resp_h, c$id$resp_p);
    print fmt("Informacion del flujo coincidente  IPo: %s , Po: %s , IPd: %s , Pd: %s ", p$id$orig_h, p$id$orig_p, p$id$resp_h, p$id$resp_p);
}

## Funcion auxiliar para mostrar la informacion de un solo flujo

function informacion_flujo(c: connection){
    print fmt("Informacion del flujo añadido IPo: %s , Po: %s , IPd: %s , Pd: %s, uid: %s ", c$id$orig_h, c$id$orig_p, c$id$resp_h, c$id$resp_p, c$uid);
}

\end{lstlisting}

\noindent La función informacion\_flujo sirve para mostrar la información del flujo 
relativa a las IP’s y a los puertos que contiene, así como el \textit{UID} 
del flujo, que es el identificador único que se le asigna automáticamente 
a cada flujo. 
\intro
Por su parte la función informacion\_coincidencia sirve para mostrar la 
información de dos flujos a la vez, mostrando sus IP’s y sus puertos. 
Esta función será usada para mostrar la información de los dos flujos que 
son emparejables, siendo el primero de ellos el que se usa para comparar siempre 
y el segundo es el nuevo flujo que se ha detectado. De esta forma tan sencilla 
se puede mostrar toda la información interesante sin tener que recorrer al final de 
la ejecución del programa todo la \textit{tabla de connection}.

\intro
Para eliminar los flujos que, por su \textit{timestamp} estén a punto de morir, Bro nos 
da la opción mediante un evento de gestionar que hacer antes de que sean eliminados 
de la memoria, dicho evento es: 

\begin{itemize}
\item connection\_state\_removed, evento que se lanza cuando un flujo activo va a ser 
eliminado de la memoria.
\end{itemize}

\noindent Cabe destacar que el borrado no es todavía operativo, por lo que en el script puede 
que este evento esté comentado.
\intro
\begin{lstlisting}[style=Codigoc]
## Generated when a connection’s internal state is about to be removed from memory. Bro generates this event reliably
## once for every connection when it is about to delete the internal state. As such, the event is well-suited for
## script-level cleanup that needs to be performed for every connection.
## This event is generated not only for TCP sessions but also for UDP and ICMP flows.

event connection_state_remove(c: connection){

  local orig = c$id$orig_h;
  local dest = c$id$resp_h;
  local po = c$id$orig_p;
  local pd = c$id$resp_p;
  local coleccion = collection[orig,dest,po,pd];
  local tama=|coleccion|;


  for(j in coleccion){
    if(j+1 >= tama){
      if(tama==1){
        collection[orig,dest,po,pd]=vector();
      }
      if(j+1==tama){
        delete collection[orig,dest,po,pd][tama]; ## Aqui esta el error
      }
      break;
    } else {
      collection[orig,dest,po,pd][j]=coleccion[j+1];
    }
  }

  print fmt("Terminamos copia y borrado...");

}

\end{lstlisting}

\noindent En este evento lo que tratamos de hacer es, antes de eliminarlo de 
la memoria, ver si existe algún flujo que pueda ocupar su lugar 
siendo el flujo referencia a la hora de comparar flujos. Básicamente 
lo que hace es comprobar si el tamaño es mayor que 1. Si es 1 solamente 
tendremos que borrar el vector contenido en la posición de la tabla, 
y si tenemos que es mayor, tendremos que borrar el primero y mover todos 
una posición hacia atrás. Nos interesa que este evento funcione, pues 
si los \textit{timestamps} de dos flujos tienen mucha diferencia los daremos 
como que no son comparables, cuando si se van actualizando, tendremos más 
posibilidades de que la clasificación sea correcta.
\intro
Para añadir flujos a la tabla de conexiones, tenemos un evento 
dedicado a ello:
\intro
\begin{itemize}
\item new\_connection, Bro lanza este evento cada vez que detecta un flujo que era desconocido.
\end{itemize}

\begin{lstlisting}[style=CodigoC]
## Cada vez que entra un nuevo flujo compruebo que si esta en la tabla
## Este evento se lanza con cada nueva conexion de un flujo que no sea conocido, pero al borrarlo de memoria será desconocido aunque ya lo tengamos en la tabla
## Generated for every new connection. This event is raised with the first packet of a previously unknown connection. Bro uses a flow-based definition of “connection” here that includes not only TCP sessions but also UDP and ICMP flows.

event new_connection(c: connection){

  local orig = c$id$orig_h;
  local dest = c$id$resp_h;
  local po = c$id$orig_p;
  local pd = c$id$resp_p;
  print fmt("new_connection");
 
  if( [orig,dest,po,pd] !in collection ){

    ## Si no estan los valores clave del flujo lo creamos
    collection[orig,dest,po,pd]=vector(c);
    informacion_flujo(c);
    print fmt("Añadimos una nueva conexion");
  
  } else {

    ## Si ya esta, lo añadimos
    collection[orig,dest,po,pd][|collection[orig,dest,po,pd]|] = c;
    informacion_flujo(c);
    print fmt("Ya esta y la añadimos");
 
  }

}
\end{lstlisting}

\noindent Lo que se realiza en este evento es ver si en la tabla global tenemos 
ya almacenado un flujo de esas características. Si no tenemos uno igual 
creamos en la tabla para los valores de las IP's origen, destino y puertos 
de origen y destino un nuevo vector que contenga solamente este nuevo flujo. 
En caso de que si tengamos los valores ya almacenados solamente tendremos que 
añadir al final del vector el nuevo flujo. En ambos casos mostraremos la 
información del flujo.
\intro
Por último usamos dos eventos en el programa para ver cuando se 
lanza y cuando finaliza el programa:
\intro
\begin{itemize}
\item bro\_init, evento que se lanza cuando iniciamos Bro.
\item bro\_done, evento que se lanza cuando Bro va a terminar.
\end{itemize}
\intro
\begin{lstlisting}[style=CodigoC]
## Evento que se lanza cuando se inicia BRO

event bro_init(){

  print fmt("Hora de inicio: %s", current_time());

}

## Evento que se genera cuando BRO va a tenerminar

event bro_done(){

 
  print fmt("Hora de finalizacion: %s", current_time());

}
\end{lstlisting}

\noindent En el primer evento, que es el que se lanza cuando Bro se ejecuta, 
mostramos la hora de inicio. En el segundo evento, que solamente 
se ejecuta cuando Bro va a finalizar, mostramos la hora de finalización.
