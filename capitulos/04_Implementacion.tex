\chapter{Implementación}

Ahora vamos a proceder a explicar cómo se ha solucionado los problemas 
propuestos, siendo estos explicados a través del programa que se ha 
realizado, quedando así más clara la explicación.
\intro
\begin{verbatim}
## funcion para la comparacion de los flujos, c1 el flujo que esta en el set conex y c2 para el flujo que es candidato a guardarse en empa
function emparejamiento(c1: connection, c2: connection ):double {

  local Nip=1; ## Variable para saber cuantas conexiones tenemos
  local Po1: count; ## Puerto origen del primer flujo
  local Po2: count; ## Puerto origen del segundo flujo
  local Pd1: count; ## Puerto destino del primer flujo
  local Pd2: count; ## Puerto destino del segundo flujo
  local k1 = 1;  ## Variable fija
  local k2 = 10; ## Variable fija
  local dt: double; ## Variable para la diferencia de los tiempos
  local resultado = 0.0; ## Lo ponemos a 0
  print c1$uid;
  print c2$uid;
## Podemos saltarnos este bucle si inicializamos Nip a 1
  ## for (s in conex){

  ##   if((s$id$orig_h == c1$id$orig_h) && (s$id$resp_h == c1$id$resp_h) && (s$id$orig_p == c1$id$orig_p) && (s$id$resp_p == c1$id$resp_p)){
  ##           Nip=Nip+1;
  ##           print fmt("Numero de Nip sin table: %d", Nip);
  ##           break;
  ##   }
  ## }

  if(c1$uid==c2$uid){
    print fmt("Son el mismo flujo, no se realiza incremento en Nip");
  }else{
## Este bucle lo puedo hacer sin ningun problema, pues en los eventos todavia no se ha dicho que se guarde en el set
  for (i in empa){
    if((i$id$orig_h == c2$id$orig_h) && (i$id$resp_h == c2$id$resp_h) && (i$id$orig_p == c2$id$orig_p) && (i$id$resp_p == c2$id$resp_p)){
            Nip=Nip+1;

    }
  }
  print fmt("Numero de Nip en table: %d", Nip);
  informacion_coincidencia(c1,c2);
  print fmt("Tiempo de inicio del flujo: %s", |c1$start_time|);
  print fmt("Tiempo de inicio del flujo: %s", |c2$start_time|);
  ## Para dp1 y dp2 que son 1-norm usamos la "Manhattan norm" que dice lo siguiente: SAD(x1,x2) = sumatoria(x1i - x2i)
  ## k1 y k2 son dos variables que nosotros le ponemos de forma manual, en este caso las pondremos como locales con 1 y 10 respectivamente
  ## dt es la diferencia de tiempo entre los time stamp de los primeros flujos de los flujos
  ## el tipo time se supone que es como un double, por lo tanto podremos restarlos sin problemas
  ## para la comparacion de puertos primero tendremos que hacer uso de la funcion  port_to_count [https://www.bro.org/sphinx/scripts/base/bif/bro.bif.bro.html#id-port_to_count]
  ## la cual nos pasa el puerto, que recordamos que va tambien con un string en el cual se nos dice que tipo es, a un
  ## valor numerico que si podremos restar sin problemas
  ## La funcion quedaria asi: (Nip-1)+(1/(dp1+k1))+(1/(dp2+k1))+(1/(dt+k2))
  Po1=port_to_count(c1$id$orig_p);
  Pd1=port_to_count(c1$id$resp_p);
  Po2=port_to_count(c2$id$orig_p);
  Pd2=port_to_count(c2$id$resp_p);
  ## local t1: double;
  ## local t2: double;
  ## t1 = time_to_double(c1$start_time);
  ## t2 = time_to_double(c2$start_time);

  dt=(|c1$start_time| - |c2$start_time|);

  ## print fmt("Tiempo paquete 1: %s", t1);
  ## print fmt("Tiempo paquete 2: %s", t2);
  print fmt("Diferencia de tiempo: %s", dt);
  resultado=(Nip-1)+(1/((Po1-Po2)+k1))+(1/((Pd1-Pd2)+k1))+(1/(dt+k2));
 }
 return resultado;

}
\end{verbatim}
En este trozo de código se muestra la función que se usa para devolver 
el resultado de la función de comparación. El resultado será comparado 
con el umbral que nosotros definimos, haciendo que si dicho resultado 
es mayor que el umbral serán emparejables y por tanto serán guardados 
en un table[connection] of connection, lo que sería comparable en C++ 
con un map. Decidimos guardarlos dentro de cada uno de los eventos que 
lanzamos cuando Bro reconoce un determinado protocolo, dichos eventos son:
\intro
\begin{itemize}
\item connection\_established, el cual se lanza cuando Bro localiza un SYN-ACK que corresponda a un handshake de TCP.
\item connection\_finished, el cual lanza Bro cuando una conexión TCP finaliza de forma normal. 
\item udp\_request, el cual se lanza cuando Bro localiza un flujo que corresponde con UDP lanzado desde el origen.
\item udp\_reply, que es lanzado por Bro cuando localiza una respuesta a un flujo del anterior evento.
\item icmp\_echo\_request, lanzado por Bro cuando localiza un flujo de tipo ICMP de echo.
\item icmp\_echo\_reply, el cual es lanzado por Bro cuando localiza una respuesta a los flujos del evento anterior.
\end{itemize}

Dentro de estos eventos descritos se realiza la comprobación de que dos 
flujos se pueden comparar, ya que tienen los mismos IP’s y puertos de 
origen y destino. Una vez en la función de comparación se almacenarán 
dichos puertos, pasándolos al tipo count para su uso en la función, se 
establecerán dos constantes, se calculará la diferencia de tiempo y en 
una variable se almacenará el número de flujos que hay con esos datos. 
Cuando realicemos esto, aplicaremos la función de cálculo y podremos 
decir si los flujos son emparejables.
\intro
Para mostrar la información de los flujos tenemos dos opciones, de las 
cuales mostramos el código a continuación:
\intro
\begin{verbatim}
## Creo funcion auxiliar para ver la informacion del flujo nuevo que se añade, no de todos los flujos todo el rato
function informacion_flujo(c: connection){
    print fmt("Informacion del flujo nuevo IPo: %s , Po: %s , IPd: %s , Pd: %s ", c$id$orig_h, c$id$orig_p, c$id$resp_h, c$id$resp_p);
}


## Creo funcion auxiliar para ver la informacion del flujo que se coincide
function informacion_coincidencia(c: connection, p: connection){
    print fmt("Informacion del primer flujo  IPo: %s , Po: %s , IPd: %s , Pd: %s ", c$id$orig_h, c$id$orig_p, c$id$resp_h, c$id$resp_p);
    print fmt("Informacion del flujo coincidente  IPo: %s , Po: %s , IPd: %s , Pd: %s ", p$id$orig_h, p$id$orig_p, p$id$resp_h, p$id$resp_p);
}

La función informacion_flujo sirve para mostrar la información del flujo relativa a las IP’s y a los puertos que contiene. Por su parte la función informacion_coincidencia sirve para mostrar la información de dos flujos a la vez, mostrando sus IP’s y sus puertos, la cual será usada para mostrar al final la información que tenemos en la tabla y el flujo con el que coincide en el set de comparaciones.

En el programa se hace uso también de un evento para eliminar los flujos que estén a punto de morir, este evento es:

connection_state_removed, evento que lanza Bro cuando va a eliminar de la memoria un flujo activo.

## Cuando la conexion va a ser borrada la eliminamos del set y en caso de tener otra conexion en el empa la añadimos
## se obtienen los mismos flujos añadidos que eliminados, por lo tanto hay que controlar cuando lo añadimos y cuando lo eliminamos
## Sirve para TCP, UDP e ICMP
## Generated when a connection’s internal state is about to be removed from memory. Bro generates this event reliably
## once for every connection when it is about to delete the internal state. As such, the event is well-suited for
## script-level cleanup that needs to be performed for every connection.
## This event is generated not only for TCP sessions but also for UDP and ICMP flows.
event connection_state_remove(c: connection){

## Creo un connection local para poder pasarlo de empa a conex
   local cl: connection;
## Variable booleana para controlar el acceso al set
   local esta = F;

## for que va recorriendo el set y haciendo comparaciones
    for(s in empa){
      if((s$id$orig_h == c$id$orig_h) && (s$id$resp_h == c$id$resp_h) && (s$id$orig_p == c$id$orig_p) && (s$id$resp_p == c$id$resp_p)){
## Si se dan todas las condiciones la variable booleana de control de acceso al set se cambia a true, T
              esta=T;
## Al existir otro flujo lo copiamos en cl
              cl=s;
              break;
      }
    }

    ## Aqui si tenemos otro flujo igual al que vamos a eliminar lo metemos en conex para que ocupe el lugar del que vamos a borrar
    ## Con la variable booleana controlamos el decrecimiento del set
    if (esta==T){
      delete conex[c];
      add conex[cl];
      delete empa[cl];
      ## print fmt("Hemos borrado");
      ## print empa[cl];
    } else {
      delete conex[c];
    }
    elimi=elimi+1;
    ## Quitamos uno al tamaño del set
    tams=tams-1;
    esta=F;
    ##  print fmt("Tamanio del set: %d", tams);
    ##  informacion_flujo(c);

}
\end{verbatim}

En nuestro caso antes de eliminarlo del set de comparaciones 
lo que hacemos es ver si en la tabla de emparejados existe un 
flujo con los mismos datos, en caso de que exista borramos el 
flujo del set, pasamos al set el flujo de la tabla de emparejados 
y este flujo lo eliminamos de la tabla. La ventaja que tiene 
esto es que el timestamp será actualizado, haciendo que la 
diferencia de los tiempos de dos flujos no sea demasiado grande 
y por lo tanto teniendo más posibilidades de que exista algún 
flujo que sea emparejable con el nuevo. En caso de que no exista 
un flujo con las características deseadas para la sustitución en 
el set de conexiones, simplemente será borrado, si no ha tenido 
ningún flujo con el cual ha sido emparejado en todo el uso del 
programa, este flujo borrado nunca será mostrado, en caso de que 
si lo tuviese, el flujo siempre estará almacenado en una tabla 
que mostramos al finalizar el programa.
\intro
Para añadir flujos al set de conexiones, tenemos un evento 
dedicado a ello:
\intro
\begin{itemize}
\item new\_connection, Bro lanza este evento cada vez que detecta un flujo que era desconocido.
\end{itemize}

\begin{verbatim}
## Cada vez que entra un nuevo flujo lo comparo con lo que ya tengo en el set
## Este evento se lanza con cada nueva conexion de un flujo que no sea conocido
## Generated for every new connection. This event is raised with the first packet of a previously unknown connection. Bro uses a flow-based definition of “connection” here that includes not only TCP sessions but also UDP and ICMP flows.
event new_connection(c: connection){

## Si el set esta vacio meto el primer flujo
   if(tams==0){
    add conex[c];
   }
## Sumamos uno al tamaño del set
    tam=tam+1;

## Variable booleana para controlar el acceso al set
     local met = F;

## for que va recorriendo el set y haciendo comparaciones
     for(s in conex){
## Copiamos en la variable local para comparar con todo lo que hay en el set
       if((s$id$orig_h != c$id$orig_h) && (s$id$resp_h != c$id$resp_h) && (s$id$orig_p != c$id$orig_p) && (s$id$resp_p != c$id$resp_p)){
               ## Si se dan todas las condiciones la variable booleana de control de acceso al set se cambia a true, T
               met=T;
       }

     }
    ## Con la variable booleana controlamos el crecimiento del set
     if (met==T){
       add conex[c];
       tams=tams+1;
       ## print fmt("Meto un flujo nuevo por la conexion de origen distinta");
     }
     met=F;
    ## print fmt("Numero de flujos al momento: %d", tam);
    ## print fmt("Tamanio del set: %d", tams);
    ## informacion_flujo(c);

}
\end{verbatim}

Lo primero que debemos de hacer es ver si el tamaño del set de 
conexiones es 0, en caso de serlo añadimos el flujo directamente, 
en caso de que no sea 0 tenemos que ver primero si tenemos un flujo 
con los datos de comparación iguales, en caso de tenerlo el flujo 
no será añadido, en caso de que no exista otro flujo con los mismos 
datos de comparación lo guardaremos.
\intro
Por último usamos dos eventos en el programa para ver cuando se 
lanza y cuando finaliza el programa:
\intro
\begin{itemize}
\item bro\_init, evento que se lanza cuando iniciamos Bro.
\item bro\_done, evento que se lanza cuando Bro va a terminar.
\end{itemize}
\intro
\begin{verbatim}
## Evento que se lanza cuando se inicia BRO.
event bro_init(){

  print fmt("Hora de inicio: %s", current_time());

}

## Evento que se genera cuando BRO va a tenerminar, menos si se realiza mediante una llamada a la funcion exit (ver documentacion)
event bro_done(){

  ## Mostramos lo que tenemos en la tabla de emparejados
  for(s in emparejados){
    ## print fmt("Tamaño de la fila de la tabla: %d", |empa[s]|);
    ## print fmt("Tenemos: %s en %s a %s en %s", emparejados[s]$id$orig_h, emparejados[s]$id$orig_p, emparejados[s]$id$resp_h, emparejados[s]$id$resp_p);
    ## print fmt(" de %s en %s a %s en %s", s$id$orig_h, s$id$orig_p, s$id$resp_h, s$id$resp_p);
    informacion_coincidencia(emparejados[s], s);
  }

  ## for(i in emparejados){
    ## print fmt("Tenemos lo siguiente:");
    ## print emparejados[i];
  ## }

  print fmt("Total de flujos: %d", tam);
  print fmt("Hora de finalizacion: %s", current_time());
}
\end{verbatim}

En el primer evento solamente mostramos la hora a la que se 
inicia el programa mediante la función current_time. En el 
evento del final mostramos los flujos que han sido emparejados 
y mostramos la hora de finalización del programa.
