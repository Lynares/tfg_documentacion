\chapter{Implementación}\label{implementacion}

A continuación, se expondrá cómo se ha implementado el módulo de Bro, el cual sirve como resolución al problema planteado en 
\ref{sec.emparejamiento}.

\intro La descripción del módulo se realizará mostrando las cabeceras de las funciones y eventos empleados, comentando después para 
que sirve cada uno. De forma que se tendrá una descripción de la implementación a modo de \textit{API}.

\section{Creación del registro}

La creación del registro sobre el que se escribirán los flujos que son emparejables, se realiza mediante un evento.

\begin{lstlisting}[style=CodigoC]
event bro_init()

\end{lstlisting}

\intro Este evento es lanzado cuando Bro se inicia. Cuando se inicie se creará un registro, en el que se escribirán las IP's de origen 
y destino, así como los puertos. Además, se el \textit{uid} de cada flujo, para poder realizar búsquedas rápidas. También, entre la 
información de los flujos habrá un string, cuya función es separarlos para una mejor visibilidad.

\section{Detección de flujos nuevos}

A continuación, se explicará el evento encargado de gestionar los flujos de nuevos.

\begin{lstlisting}[style=CodigoC]
event new_connection(c: connection)

\end{lstlisting}

\intro Este evento recibe como entrada una nueva conexión, la cual no está identificada previamente. Esto quiere decir que es nueva
o que ha sido borrada de la memoria y ha vuelto a ser detectada.

\intro Una vez que se detecte el flujo, se deberá de comprobar si existe, de forma que se descartará, o si no existe, por lo que se 
creará un nuevo índice en la tabla de activos y se almacenará.

\intro Este tipo de evento detecta las conexiones de tipo \textit{TCP y UDP}.


\section{Eventos de detección de tráfico}
\intro A continuación, se verán los distintos eventos que van a detectar el diferente tipo de tráfico.

\begin{lstlisting}[style=CodigoC]
event connection_established(c: connection)

event connection_finished(c: connection)

event udp_request(u: connection)

event udp_reply(u: connection)

\end{lstlisting}

\intro Los dos primeros eventos son los correspondientes al tráfico \textit{TCP}. Los otros dos, como se indica  
en el nombre están destinados al tráfico \textit{UDP}.

\intro El primer evento relacionado con \textit{TCP}, se activa cuando se detecta un paquete \textit{SYN-ACK} que 
responde al \textit{handshake} que se realiza en las conexiones de este tipo.

\intro El segundo evento detecta cuando la conexión \textit{TCP} finaliza de forma normal.

\intro Los dos eventos relacionados con \textit{UDP} detectan paquetes de dos tipos distintos.

\begin{itemize}
\item \textit{UDP request}. Se genera por cada paquete que es enviado por el creador del flujo.
\item \textit{UDP reply}. Este es generado por cada paquete que es enviado por el receptor del flujo.
\end{itemize}

\intro Estos dos últimos eventos son bastantes costosos, pero son absolutamente necesarios, ya que son los únicos 
eventos que detectan este tipo de conexión.

\intro Dentro de todos estos eventos se realiza el mismo proceso. Primero se obtienen las IP's y los puertos del flujo y se obtiene 
el flujo activo correspondiente a ese flujo. Después de esto, se comprueba que IP y que puerto coinciden, llamando así a la función de 
\textit{calculo} para que realice el filtrado.

\intro Aparte de estos eventos, también existen dentro del módulo otros dos eventos, destinados a detectar 
paquetes del protocolo \textit{ICMP}. Estos eventos son necesarios, pues Bro los detecta igual que los 
de tipo \textit{TCP y UDP}, aunque no pertenezca a la capa de transporte.

\begin{lstlisting}[style=CodigoC]
event icmp_echo_request(c: connection, icmp: icmp_conn, id: count, seq: count, payload: string)

event icmp_echo_reply(c: connection, icmp: icmp_conn, id: count, seq: count, payload: string)
\end{lstlisting}

\intro El protocolo \textit{ICMP} se usa para el control, enviando mensajes de error en caso de que un router o un host 
no sean accesibles.

\intro Al igual que con los eventos de \textit{UDP}, el \textit{request} es enviado por el creador del flujo, 
siendo una petición. El \textit{reply} es enviado por el receptor del \textit{request}, por lo que se considera 
la respuesta del anterior. El funcionamiento es el mismo que en los anteriores eventos que gestionan el tráfico.

\section{Procesado de flujos y escritura en el registro}

Ahora, se va a describir cómo se procesan los flujos antes de ser enviados a comparar. A su vez, se hablará de cómo se escribe en el 
registro.

\begin{lstlisting}[style=CodigoC]
function calculo(c1: connection, c2: connection )
\end{lstlisting}

\intro Esta función se usa para procesar los flujos antes de llamar a la función de emparejamiento. Aquí, se comprueba primero que no se está analizando el mismo flujo. Una vez que esto ha sido descartado, se llama a la función de emparejamiento, y con el número que devuelve, se comprueba si son emparejables o no. Si se obtiene un resultado mayor que el umbral, serán emparejables, y por lo tanto se almacenarán en la tabla de emparejados. En caso contrario, no se almacenarán en la tabla de emparejados.

\intro Además de lo ya detallado, cuando son emparejables, se escribirá la información de los dos flujos en el registro.

\section{Cálculo de emparejamiento}

A continuación, se explicará cómo se aplica la ecuación \ref{ecug} en el módulo.

\begin{lstlisting}[style=CodigoC]
function emparejamiento(c1: connection, c2: connection ):double 
\end{lstlisting}

\intro Esta función recibe como entrada dos flujos y devuelve un número, de tipo \textit{double}. 

\intro Lo primero que se hace es sacar las IP's de origen y destino y los puertos de origen y destino de los flujos. También, se 
obtendrán los \textit{timestamps} de los flujos, de esta forma se conseguirá la diferencia de tiempo. 

\intro Para poder operar con los puertos, habrá que pasarlos al tipo \textit{count}, de esta forma se elimina la 
terminación con el tipo de protocolo del puerto. Se puede operar con esta terminación, pero al tener que usar después otros tipos de 
datos que no son puertos es mejor quitar la terminación, pues de lo contrario se obtendrá un resultado erróneo. 

\intro Lo mismo que con los puertos pasa con el tipo \textit{time}. Se puede operar con este tipo de dato pero no es recomendable 
hacerlo ya que se va a trabajar con otros datos de carácter matemático.

\intro El número de IP's coincidentes en ambos flujos, o \textit{$N_{IP}$} en la fórmula, es de fácil cálculo, pues bastará con saber 
si coinciden una IP o las dos.

\section{Borrado de flujos}

Ahora, se describirá el funcionamiento del borrado de flujos.

\begin{lstlisting}[style=CodigoC]
event connection_state_remove(c: connection)

\end{lstlisting}

\intro Este evento se activa cuando el flujo que entra como parámetro va a morir, o ser borrado de la memoria. Es un flujo 
que ya ha sido procesado por el módulo.

\intro Lo que se realiza dentro de este evento es buscar en la tabla el índice el vector. De esta forma 
se borrará el primer flujo almacenado en el vector.

\intro De ser el único flujo almacenado en el vector, se borrará el vector entero. Si hay más flujos almacenados, se moverán una 
posición hacia atrás. De esta forma, se seguirá teniendo un rendimiento óptimo al no tener otra tabla que indique que flujos son 
borrados.

\section{Almacenamiento}

\intro Por último, se verá cómo se ha implementado las estructuras de almacenamiento de los flujos.

\begin{lstlisting}[style=CodigoC]
global activos: table[addr, port] of vector of connection;

global emparejados: table[addr, port] of vector of connection;
\end{lstlisting}

\intro Se trata de dos tablas globales, cuyo indice está constituido por una IP, ya sea la de origen o la de destino, un puerto. 
Además, como los indices son únicos, cada índice apuntará a un vector, y dentro de este se almacenarán los flujos ordenados, en 
función de cuando son detectados.

\intro La primera tabla será usada para almacenar los flujos que están activos, por lo tanto el primer flujo de cada vector será el 
usado para realizar las comparaciones. La segunda se empleará para almacenar los que ya están emparejados, por lo que contiene flujos 
que pueden haber sido borrado de memoria.
