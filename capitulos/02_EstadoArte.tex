\chapter{Estado del arte}

Aquí se encontrará la parte teórica del proyecto. Se desglosará en que consiste la identificación de tráfico y 
las distintas técnicas que existen. También se hablará de Bro \cite{broindex}, el NMS que se usará para 
llevar a cabo el desarrollo del módulo. Se introducirá en que consiste, cómo gestiona los eventos y sus 
funcionalidades, así como la posibilidad de ampliar estas.

\intro Por último se presentará de forma teórica en que consiste el emparejamiento de flujos y de qué forma 
se podría usar para identificar el tráfico.

\section{Identificación de tráfico}

La identificación de tráfico, por si sola, no clasifica el tráfico. Solamente proporciona información obtenida 
de los paquetes. Esta información puede ser obtenida con distintas técnicas, las cuales se detallarán a 
continuación.

\subsection{Técnicas de identificación de tráfico}

Existen varias técnicas de identificación de tráfico, cómo se pudo ver en el capítulo de introducción. Ahora 
serán explicadas de una forma más detallada.

\intro Clasificación basada en los puertos de la capa de transporte. \cite{iana}

\intro Esta clasificación consiste en clasificar el tráfico dependiendo de los puertos de la capa de transporte, 
fue diseñada por IANA \cite{ianaexplicacion}. IANA es la organización que se encarga de asignar los protocolos 
de Internet, algunos nombres de dominios y coordinar la asignación de direcciones IP. 

\intro Entonces, IANA es quien asigna el número de puerto a los distintos protocolos, haciendo que se pueda 
clasificar el tráfico por el número de puerto al que se envía. Este sistema tiene el inconveniente de que 
ahora mismo, se están desarrollando multitud de aplicaciones web, las cuales se conectan mediante el 
puerto 80, es decir, mediante el protocolo HTTP. Pero esto no es cierto, pues se puede hacer que cualquier 
aplicación envíe información por el puerto 80, sin que sea HTTP, lo cual daría una mala clasificación.


\subsection{Identificación de tráfico basada en flujos}

Clasificación basada en la aplicación de técnicas de aprendizaje automático sobre estadísticas 
de tráfico \cite{learning}, se aplica sobre flujos.

\intro Esta clasificación es muy sencilla. Consta de algoritmos de aprendizaje automático, los cuales se 
dedican a realizar análisis estadísticos. De esta forma aprenden las características determinadas de 
los flujos. Este tipo de clasificación tiene un porcentaje de acierto del 95\%, según se puede leer en el 
artículo \cite{comparacion}.

\subsection{DPI}

Clasificación basada en el contenido del paquete o DPI. \cite{payload}

\intro Esta técnica de clasificación consiste en analizar los paquetes, más allá de su cabecera. La Inspección Profunda 
de Paquetes, DPI por sus siglas en inglés, \textit{Deep Packet Inspection}. Esta técnica lo que hace es 
analizar todos los paquetes en busca de cadenas que se repitan, además de las IP's y los puertos. Una vez que 
tiene una cadena, la usa para buscar paquetes que tienen ese patrón, de esta forma establece la clasificación.

\intro Esta técnica es útil para buscar virus, intrusiones y demás. Suele ser usada por los proveedores de 
servicios de Internet, ISP, por sus siglas en inglés \textit{Internet Service Provider} y grandes empresas. 
Se puede decir que es una técnica que no respeta la privacidad, pues entra dentro de la información del paquete, 
más allá de la cabecera. Aunque en el caso de una gran empresa si puede hacerse, pues si se está en la red propia 
no es delito mirar la información. En algunos países, como Alemania, donde es ilegal el uso de \textit{torrents}, 
las autoridades pueden pedir información al proveedor de servicios sobre uno de sus clientes, si cree que este 
está infringiendo la red, por lo que esta técnica es ideal para este tipo de cometidos.

\intro Esta técnica también es ideal para saber si se está siendo atacado. \cite{dpiaproximacion}

\section{BRO}

\subsection{Funcionalidades básicas de BRO}

\subsection{Eventos y trazas}

\subsection{Incorporación de funcionalidades}

\section{Emparejamiento de flujos}

El emparejamiento de flujos, por si mismo, sólo identifica el tráfico, por lo tanto se necesitará otra técnica 
para clasificar el tráfico.

AQUI EMPIEZA LO ANTIGUO, REHACER
\intro
La idea de la que partimos es la siguiente: 
Los flujos son un conjunto de paquetes que comparten la misma 
información, IP de origen, IP de destino y los puertos de origen 
y destino, aparte de esta información también tienen un tiempo de 
inicio, relativo al \textit{timestamp} del primer paquete del flujo y una 
duración. Con estos datos ya podemos empezar a trabajar en el 
emparejamiento, pues si dos flujos comparten las IP’s de origen y 
destino y los puertos de origen y destino ya serían candidatos a 
la comparación. 
\intro
Para comparar los flujos lo que haremos será aplicar la siguiente 
fórmula:
\intro

\begin{equation*}
	F(x,y)=
 	\begin{cases}
	  G(x,y), & NIP(x,y) \geq 1 \\
	  -\infty, & \text{en otro caso}
	 \end{cases}
\end{equation*}

\noindent Donde tenemos que G(x,y) es:
\intro
\begin{displaymath}
G(x,y) = |NIP(x,y) – 1| + 1 / (dp1(x,y) + k1) + 1 / (dp2(x,y) + k1) + 1 / (dt(x,y) + k2)
\end{displaymath}

\noindent Tenemos que \textbf{x} e \textbf{y} son los flujos a comparar, \textbf{NIP(x,y)} es el número 
de veces que tenemos almacenado flujos similares, por lo cual siempre 
será uno, como mínimo. \textbf{dp1(x,y)} son los puertos de origen de los dos 
flujos, que al comparar siempre flujos que tienen los mismos puertos serán siempre el mismo puerto.
\textbf{dp2(x,y)} son los puertos de destino, que como en el caso anterior siempre serán el mismo puerto.
\textbf{k1} y \textbf{k2} son constantes definidas antes de ejecutar el programa, 
estando estas normalmente en los trabajos realizados anteriormente estas constantes suelen estar 
definidas entre 1 y 10000 y, por último, \textbf{dt} es la diferencia de tiempo entre 
los \textbf{timestamps} de los dos flujos.
\intro
Con esta fórmula obtendremos un número que podremos comparar con un 
umbral que estará definido a la hora de ejecutar el programa, si el umbral 
definido es pequeño tendremos más flujos almacenados, si el umbral es más grande serán 
menos los flujos que serán emparejables y por tanto almacenados para trabajar con ellos. \cite{comparacion}




\section{NMS y BRO}

Bro es un NMS, Network Monitoring System, cuya principal actividad 
es analizar el tráfico de una red para la búsqueda de amenazas. Una de 
las muchas peculiaridades de Bro es que usa un lenguaje propio, llamado 
también Bro, el cual tiene similitudes con \textit{Python} y \textit{Java}, y se gestiona 
mediante \textbf{eventos}. En el caso de este trabajo se utilizará dicho lenguaje, 
el cual en un principio puede parecer difícil, pues no disponemos de más 
documentación que la propia de la web de Bro \cite{broindex}, pero que acaba siendo 
bastante útil para la tarea que queremos, pues tiene un tipo de dato 
llamado \textit{connection}, el cual está relacionado con los flujos, siendo al 
final este tipo el que representa al propio flujo. En este tipo se establecen bastantes 
datos que nos serán útiles, como la IP de origen, la de destino, así 
como los puertos y sus protocolos, también guarda la información de la 
duración así como el \textit{timestamp} y más información que, en este caso, no 
será relevante, como la \textit{historia del flujo}, donde se nos informa del 
tipo de paquetes que contiene e incluso el \textit{número de paquetes} de los que consta el flujo.
\intro
En el caso de los eventos, también nos facilitarán el trabajo, pues tenemos 
eventos que gestionan todo lo que tiene que ver con los flujos, desde 
cuando aparece un nuevo flujo el cual no tenemos identificado previamente, 
lo cual para nuestro trabajo es fundamental, pues nos hace el programa más 
eficiente, ya que cada vez que se lanza ese evento significa que tenemos 
que añadir ese flujo a la lista de flujos comparables, sin tener que 
compararlo con los que tengamos previamente almacenados. También tenemos 
flujos que nos indicarán cuando un flujo esté a punto de morir, siendo el 
momento de eliminarlo de la lista de comparables y, en caso de tener 
flujos con los que éste ha sido comparado, pasará el primero de estos a 
reemplazar el flujo que va a morir en la lista de comparables, de ésta 
forma nos evitamos tener que estar comprobando la duración del flujo 
para eliminarlo manualmente. Con todo ésto conseguimos un programa eficiente 
en cuanto a tiempo y que, gracias a estos eventos, va a identificar los dos 
protocolos de la capa de transporte, siendo estos \textbf{TCP} y \textbf{UDP}. Cabe destacar que los 
eventos de \textit{UDP} empleados en nuestro programa son muy costoso, por lo tanto 
en caso de tener mucho tráfico \textit{UDP}, siendo de dos tipos distinto \textbf{reply} y 
\textbf{request}, puede hacer que el análisis tarde más tiempo del deseado, pero 
el programa suele funcionar en una horquilla de tiempo bastante aceptable 
en términos de rendimiento para el análisis del tráfico, aunque cabe 
destacar que en nuestro caso analizamos archivos, no tráfico en tiempo real. 
En el apartado \textit{Implementación} se hablará sobre los eventos usados con más detalle.
\intro
Bro también nos otorga una gran cantidad de \textbf{logs} en los cuales él mismo puede 
llegar a clasificar el tráfico dependiendo del protocolo que se use, siendo 
algunos ejemplos, \textit{HTTP, DNS, SSL, etc.} pero no nos interesa, pues se tratan de protocolos 
de la capa de aplicación. Aunque como en nuestro caso usamos la información contenida en 
un archivo \textbf{pcap}, la información de los protocolos de la capa de aplicación, 
dependerá de si han sido cargados los \textit{scripts} correspondientes cuando se 
realizó la captura del tráfico, pues en los flujos 
de Bro la información relativa a \textit{HTTP, DNS, etc.} es opcional, siendo únicamente 
obligatorio, en lo relativo a puertos que se informe de si pertenece a \textit{TCP, UDP o ICMP}.
\intro
Bro trabaja sobre Linux y sobre la terminal por lo tanto será necesario leer 
la documentación para saber que opciones nos interesa tener activadas cuando 
vayamos a analizar el tráfico, pues al no tener interfaz gráfica nos puede 
ocurrir que no tengamos activado lo que queramos y otras cosas que no 
precisamos si estén activas.
\intro
También es digno de mención que la información que hay en la red sobre Bro 
es bastante escasa, quedando casi exclusivamente la información que se puede 
encontrar en la web de Bro \cite{broindex}, por lo tanto a la hora de alguna 
duda, la web será nuestra principal opción y en caso de no encontrar lo que 
queramos tendremos que hacer bastantes pruebas hasta dar con la clave. 
También podemos realizar una búsqueda por \textit{GitHub} para coger ideas de cómo 
podemos resolver nuestras dudas. Aquí por ejemplo se puede encontrar información 
sobre cómo operar tablas con vectores de la manera correcta \cite{gitbeacon}.

\subsection{Requisitos de Bro}

Podemos descargar los binarios y el código fuente desde la web de Bro \cite{brodownload}, 
pero en mi caso después de experimentar ciertos fallos con ésta instalación, 
recomiendo seguir la instalación de Bro de su web en el apartado de instalación \cite{broinstall}, 
pues allí nos guía a una instalación que es más fácil de mantener actualizada, 
ya que descargamos el código fuente directamente del \textbf{repositorio de GitHub}, 
por lo cual si hay algún cambio importante, con alguna mejora interesante 
la podremos tener casi directamente a nuestra disposición.
\intro
Para la instalación con los archivos de \textit{GitHub}, tendremos que hacer lo siguiente:
\intro
\begin{lstlisting}[style=Consola]
sudo apt-get install cmake make gcc g++ flex bison libpcap-dev libssl-dev python-dev swig zlib1g-dev
\end{lstlisting}

\noindent Una vez que terminemos de instalar todo lo del comando anterior tendremos todos los 
prerrequisitos para la instalación de Bro, ahora clonamos el 
\textit{repositorio de GitHub} en nuestro ordenador:
\intro
\begin{lstlisting}[style=Consola]
git clone --recursive git://git.bro.org/bro
\end{lstlisting}

\noindent Cuando termine de descargarlo nos metemos en la carpeta y ejecutamos lo siguiente:
\intro
\begin{lstlisting}[style=Consola]
./configure
make
make install
\end{lstlisting}

\noindent De esta forma ya tendremos instalado Bro en nuestro ordenador y podremos empezar 
a usarlo.
\intro
Ahora para usarlo tendremos que hacer lo siguiente, lo primero será \textbf{ajustar el 
PATH} de nuestro terminal, para ello, tendremos que ejecutar en nuestro terminal 
lo siguiente:
\intro
\begin{lstlisting}[style=Consola]
export PATH=/usr/local/bro/bin:$PATH
\end{lstlisting}

\noindent Es importante hacerlo, pues sino obtendremos un error diciendo que Bro no se 
encuentra instalado. Una vez hecho esto, podemos empezar realizando un script 
sencillo y probar a ejecutarlo, un ejemplo de script sencillo es el que se ve 
en el apartado \textit{Ejemplo  de uso de evento}, 
supongamos que el script se llama \textit{prueba.bro}, para lanzarlo tendremos 
que hacer lo siguiente:
\intro
\begin{lstlisting}[style=Consola]
~$ bro -b -r pcap/nitroba.pcap scripts/bro-flows/prueba.bro
\end{lstlisting}

\noindent En este caso dentro de la carpeta Bro que se genera cuando clonamos el 
repositorio he creado una carpeta llamada scripts, para tener todos los 
scripts allí guardados, por lo tanto al encontrarnos en la carpeta raíz 
de Bro, tendremos que ponerlo en la ruta, la opción \textit{-b} que ponemos en la 
orden es solo para este ejemplo, lo que hace es no cargar los scripts 
que se encuentran en el directorio \textit{base/}. Como necesitamos leer de un 
archivo \textit{pcap}, tenemos que poner la opción \textit{-r}, que indica a Bro que tiene 
un archivo de datos que leer, al igual que con los scripts, también hemos 
creado una carpeta pcap en la cual guardamos los archivos \textit{pcap} que queramos 
leer. Una vez que lo ejecutemos se nos mostrará en el terminal un mensaje 
cada vez que se genere un flujo nuevo, cuando se exceda el timeout también 
y cuando un flujo vaya a ser eliminado de la memoria.

\subsection{Gestión de tráfico en Bro}

\subsubsection{Nacimiento y muerte de flujos}

En Bro el \textbf{nacimiento de un flujo} es controlado por un evento, dicho de otro 
modo, cuando aparece un nuevo flujo se lanza un evento, de ésta forma podremos 
controlar cuando un flujo nuevo aparece y podremos analizar toda su información 
en este evento o en los siguientes antes de que el flujo muera, o en el caso de 
tenerlo almacenado, antes de eliminarlo, pues una vez que lo eliminemos no 
podremos recuperarlo, por lo tanto para el análisis detallado del tráfico es 
interesante guardarlo.
\intro
\textbf{La muerte de un flujo} funciona también por eventos, siendo activado el evento cuando el 
flujo está a punto de morir, por lo tanto todavía podremos operar sobre él, 
por ejemplo, podremos decidir si queremos que el flujo sea eliminado o seguir 
guardándolo para análisis estadísticos, realizar una comparación para ver si tenemos más 
flujos con sus características para reemplazarlo, etc.
\intro
Por lo tanto como se comentó en apartados anteriores en Bro la mayor parte de 
la información se gestiona con eventos, lo cual facilita enormemente el trabajo 
que tenemos por delante.

\subsubsection{Marcas temporales}

Es interesante remarcar el uso de los \textbf{timestamps}, pues como ya se sabe son una 
variable importante que usaremos en la comparación, esta variable es necesaria para emparejar 
los flujos, cierto es que los datos más importantes son las IP’s y los puertos, 
pues si no son iguales no procederemos a aplicar la fórmula necesaria para el 
emparejamiento, pero llegado el momento también podríamos comparar los \textit{timestamps} 
para comprobar si son distintos, pues en caso de que sean iguales podríamos desechar la 
comparación, pues básicamente sería comparar el mismo flujo, por algún error de programación 
y esto nos haría el programa más eficiente.
\intro
En Bro, como lenguaje de programación, volvemos a encontrarnos con un tipo de 
dato propio para la gestión de estos datos, en este caso el tipo es time \cite{timetype}, 
cuyo nombre es de por si muy indicativo de lo que va a contener. La peculiaridad 
de este tipo de dato es que el tiempo viene dado en valor absoluto, por lo cual 
tendremos que formatearlo para representarlo, aunque no será necesario si solamente 
queremos operar con él. Bro dispone de funciones para pasar de \textit{double} a 
\textit{time}, \textbf{double\_to\_time}, 
obtener el tiempo actual, \textbf{current\_time}, y por último también disponemos de una 
función para obtener el tiempo del último paquete procesado, \textbf{network time}, 
obteniendo así su \textit{timestamp}, aunque parezca que necesita estar procesando de 
manera online para usar esta función no es así, pues también sirve para los 
archivos de datos de tráfico de red guardados, como los archivos \textit{pcap} que 
usaremos nosotros para el emparejamiento. Para mostrar correctamente el valor 
del tiempo podemos usar \textbf{strftime}, lo cual nos formateará el tiempo y lo mostrará 
acorde a lo que le indiquemos en la cabecera de la función. También contamos con el 
caso contrario, es decir, pasar de \textit{string} a \textit{time}, para ello tendremos que usar la 
función \textbf{strptime}.
\intro
Para acabar tenemos una de las funciones más importantes que es \textbf{time\_to\_double}, que sirve para pasar 
un tipo \textit{time} a \textit{double}, haciendo más fácil operar con él.

\subsubsection{Listado de flujos activos}

Para mantener un \textbf{listado de flujos activos} haremos uso de un contenedor para 
almacenarlos y haremos uso de los eventos de Bro para gestionarlos. Primero, cuando 
se generen flujos que previamente no habíamos visto los guardaremos para compararlos, 
si aparecen flujos que ya tenemos reconocidos, los compararemos con los almacenados 
y si nos interesa los guardaremos, y por último cuando Bro nos indique que algún flujo 
va a ser eliminado de la memoria y por lo tanto ese flujo ya no estará activo una vez que el evento termine, por lo tanto tendremos que operar con el flujo antes de que el evento termine, para ello veremos 
si hay algún candidato a ocupar su puesto en el listado de flujos emparejados y lo 
eliminamos del listado de flujos comparables.
\intro
Para mantener un listado de flujos activos Bro nos proporciona varias formas. La 
primera es usar un \textbf{vector}, el cual es fácil de usar si venimos de lenguajes como \textit{C++}, 
podremos incluir datos usando los corchetes, \textit{[ ]}, indicando el índice dentro 
de ellos, para borrar un elemento usaremos \textit{delete} y para consultar el tamaño del 
vector usaremos \textit{|v|}, pero sin embargo por si sólo es muy \textbf{ineficiente} para almacenar 
datos del tipo connection, pues solo dispone de índices numéricos.
\intro
Otro tipo para almacenar datos es \textbf{set}, el cual para borrar y consultar el tamaño 
es igual que para los vectores, sin embargo, para agregar datos tendremos que usar \textit{add se[s]}, 
estando entre los corchetes lo que queremos almacenar, aunque también podemos 
almacenar datos redifiniendo el set, por ejemplo, \textit{s1 = set(1,2,3)}. El set también 
es \textbf{ineficiente}, aunque menos, pues en un índice, aunque podemos guardar varios 
elementos estos tienen que ser únicos y corremos el riesgo de eliminar los datos almacenados al insertar nuevos datos.
\intro
Por último podemos usar un tipo \textit{table}, que es básicamente un \textit{map}, de los que se 
usan en \textit{C++}, pudiendo hacerlos bastante interesantes, pues podemos hacer un \textbf{table 
de connection de vector} o de set de connection, por lo tanto esta será la forma 
que usaremos para almacenar el listado de flujos activos, pues nos da bastante 
versatilidad a la hora de manejar el listado de los flujos, porque podemos 
acceder por índice a la lista de todos los flujos que comparten datos, haciendo 
que sea una tarea bastante asequible, en términos de eficiencia.

\subsection{Identificación de flujos en Bro (Tipo de flujo/Protocolo)}

Como ya se explicó en otro apartado, Bro consta del tipo \textit{connection} \cite{connectiontype} 
para referirse y tratar los flujos, por lo tanto cómo en el flujo hay 
información sobre el protocolo que usa, podemos consultarlo haciendo 
referencia al flujo que estamos tratando. Hay que destacar que dentro 
de \textit{connection}, en \textit{history}, existe otro tipo llamado \textit{Conn::Info} \cite{conntype} que 
extiende aún más los datos de \textit{connection}, y siendo capaz sus datos de 
crear los \textit{logs} de lo que estemos analizando, por lo que podremos generar 
un log con los protocolos que son usados en la sesión. Cabe destacar que 
mostrando la información del puerto mediante el tipo \textit{port}, el \textit{string} que 
nos muestra lleva el número del puerto y el protocolo al que pertenece, 
pudiendo así extraer si es \textit{TCP o UDP}. Aunque si lo que queremos es 
obtener solamente el protocolo de transporte al que pertenece el flujo 
podemos hacer uso de la función \textit{get\_port\_transport\_proto} \cite{getproto}.
\intro
Es destacable que el tipo de dato  \textit{port} soporta la comparación, lo cual 
será usado en el programa.  
\intro
También en las múltiples extensiones de Bro encontramos un analizador 
de protocolos, protocol analyzer \cite{protoanalyzer}, el cual dispone de varias funciones 
interesantes para el análisis de protocolos, pero que nosotros no usaremos.
\intro
Después de lo expuesto en esta sección podemos decir que a partir del protocolo al que 
nos indique el flujo que pertenece, podremos clasificar mejor el tipo de flujo que estamos 
tratando.

\subsection{Ejemplo de uso de eventos}

Ahora que tenemos algunos conceptos claros pondremos el ejemplo de un 
programa en el cual se lanza un evento cada vez que aparece un flujo nuevo.
\intro

\begin{lstlisting}[style=CodigoC]
## Para todo tipo de conexiones, ya sean TCP, UDP o ICMP. Este evento salta con cada nueva
## conexion, con el primer paquete de una conexion desconocida, por lo que mirando el
## documento generado vemos que esto es un nuevo flujo.
event new_connection(c: connection)
  {

   if ( connection_exists( c$id ) ) {
      print fmt("Nueva conexion establecida Timestamp: %s desde %s a %s", strftime("%Y/%M/%d %H:%m:%S", network_time()), c$id$orig_h, c$id$resp_h);
      print fmt("Protocolo del puerto: %s", get_port_transport_proto(c$id$orig_p));
      print fmt("Informacion de las 4 tuplas del paquete: %s", c$id);
   } else {
     print fmt("La conexion ya existe");
     print fmt("_______________________________________________________________");
   }

 }

## Solo disponible para conexiones TCP, se genera cuando no hay actividad en un
## periodo de tiempo determinado.
event connection_timeout(c: connection)
  {
    print fmt("Conexion TCP ha excedido el timeout: %s",c$id);
  }
## Este evento salta para todo tipo de conexion, se da cuando el estado interno
## esta a punto de eliminarse de memoria
event connection_state_remove(c: connection)
{

  print fmt("Conexion de %s a %s a punto de ser eliminada de la memoria", c$id$orig_h, c$id$resp_h);

}
\end{lstlisting}

\noindent Algunos comentarios sobre este código. Como habrá podido notar en el 
lenguaje \textit{Bro} los comentarios se realizan con dos almohadillas, \textit{\#\#}, 
los \textit{print} son como en la gran mayoría de lenguajes, con la particularidad 
de que si vamos a mostrar un \textit{string} debemos de indicar que va formateado 
mediante \textit{fmt}, \textit{\%s} es para mostrar el \textit{string}, tal y como se hace en \textit{C} con 
los datos de tipo \textit{int} mostrándolo con \textit{\%i}, por ejemplo. Por último para 
acceder a la distinta información que contiene el tipo de dato debemos 
de usar \textit{\$}, esto es parecido a como accedemos en \textit{Java} a los datos de las clases.
\intro
Aunque el código está comentado merece la pena hacer algunas valoraciones 
sobre los eventos usados, el evento \textbf{new\_connection} es muy útil, pues se 
ejecuta cada vez que llega un flujo que no ha sido usado antes y además 
reconoce \textit{TCP, UDP e ICMP}. El evento \textbf{connection\_timeout} aunque sólo sirva 
para flujos del protocolo \textit{TCP} es bastante útil para empezar a trabajar 
sobre Bro, pues sirve para empezar a comprender como funcionan los eventos. 
Por último el evento \textbf{connection\_state\_removed} se lanza para \textit{TCP, UDP e ICMP} 
cuando los flujos están a punto de morir, por lo tanto tiene un gran valor 
para eliminar de memoria los flujos que ya no nos interesan porque no están 
activos, y si este evento no se lanza no serán eliminados, en caso de que 
los flujos estén guardados.

