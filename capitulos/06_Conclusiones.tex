\chapter{Conclusiones y trabajo futuro}\label{conclusiones}

En este capítulo se expondrán las conclusiones alcanzadas después de realizar todo el trabajo, así como el posible trabajo futuro.

\section{Conclusiones}

Tras todo lo expuesto a lo largo de esta memoria, y tras ver los resultados en el capítulo \ref{evaluacion}, se llega a la conclusión 
de que la técnica propuesta por los investigadores del departamento de Teoría de la Señal, Telemática y Comunicaciones de la 
Universidad de Granada, es una técnica completamente funcional. Además, mejora los dos aspectos que fallan en la Inspección Profunda 
de Paquetes.

\begin{itemize}
\item \textit{Privacidad}. Este aspecto se respeta, pues los únicos datos analizados y mostrados en el registro generado, son las 
IP's, los puertos y los \textit{uids}, siendo además usado el tiempo de inicio del primer paquete del flujo. Por lo tanto, no se entra 
dentro del \textit{payload} de los paquetes, lo cual puede ser ilegal en algunos países.

\item \textit{Escalabilidad}. Se ha podido comprobar que un archivo de 1.1 GB, correspondiente al tráfico de un día entero, es 
analizado en menos de una hora siendo un tiempo aceptable, teniendo presente la cantidad de flujos que existen dentro de ese fichero. 
Un archivo más pequeño y destinado a hacer pruebas, de 56.2 MB, es analizado en un segundo.
\end{itemize}

\intro Además, los datos volcados en el registro generado son útiles y sobre ellos se puede realizar una búsqueda rápida, de modo que 
se detectaría, por ejemplo, un ataque de denegación de servicio rápidamente, pudiendo el administrador del sistema realizar las 
acciones que crea convenientes.

\section{Trabajo futuro}

Se podría continuar trabajando sobre este módulo, de forma que se haga más rápido o detecte más tipo de tráfico. Algunas de las posibles mejoras son:

\begin{itemize}
\item Añadir otras funciones, de modo que se analice el tráfico a otros niveles, como podría ser a nivel de capa de aplicación. Esto podría realizarse mediante la incorporación de eventos que detecten actividad \textit{HTTP}, por ejemplo.
\item Extender las funcionalidades ya expuestas añadiendo, por ejemplo, nuevos eventos que detecten tráfico de tipo \textit{TCP}, ya 
que este módulo solamente tiene dos funciones que detectan este tipo de tráfico, siendo el inicio y el final del protocolo lo que es 
detectado.
\item Optimizar el almacenado de flujos. Se podría optimizar el módulo mejorando la forma en que son guardados los flujos, así 
como el borrado de los mismos.
\end{itemize}