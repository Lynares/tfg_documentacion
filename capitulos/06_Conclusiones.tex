\chapter{Conclusiones y trabajo futuro}\label{conclusiones}

Por último, se expondrán las conclusiones a las que se han llegado, así como el posible trabajo futuro a partir de lo realizado.

\section{Conclusiones}

Tras todo lo expuesto a lo largo de esta memoria, y tras ver los resultados en el capítulo \ref{evaluacion}, se llega a la conclusión de 
que la técnica propuesta por los investigadores del departamento de Teoría de la Señal, Telemática y Comunicaciones de la Universidad de 
Granada, es una técnica completamente funcional. Además, se trata de una técnica que respeta la privacidad y completamente escalable.

\intro Respeta la privacidad, por que no entra dentro del payload de los paquetes. A su vez, es escalable, cómo se ha visto, pues aunque 
el archivo a analizar sea mucho mayor, los tiempos de análisis son aceptables con respecto al tamaño. Además, no sufre de ningún tipo de 
retardo.

\intro Los datos mostrados en los registros que se generan, son útiles para realizar una búsqueda rápida en el caso de que sea necesaria 
y, además, con un clasificador adecuado la tarea de la clasificación será muy llevadera.


\section{Trabajo futuro}

En el futuro se podría seguir trabajando en este módulo, añadiendole funcionalidades que de momento no tiene, por ejemplo, se podrían 
agregar más eventos que detecten tráfico \textit{TCP}, pues Bro proporciona bastantes de este tipo. 

\intro A su vez, podría identificarse tráfico de la capa de aplicación, simplemente añadiendo los módulos adecuados de Bro. Por lo tanto, hay mucho posible trabajo que hacer.
